\chapter{Zakończenie}

\section{Realizacja tematu}

W ramach projektu udało się zrealizować wszystkie zamierzone cele, czyli stworzenie wiarygodnego systemu społecznościowego opinii o~produktach. Wymaganiami określonymi przez grupę TWO były:
\begin{itemize}
\item Moduł społecznościowy
\item Moduł informacji o~produktach
\item Moduł komunikacji z~zewnętrznym serwisem
\item Moduł oceny wiarygodności
\item Moduł administracyjny
\end{itemize}
Rezultat ten został osiągnięty dzięki dobrze zorganizowanej pracy nad projektem oraz wsparciu grupy TWO i~promotora. Dodatkowym czynnikiem wpływającym na sprawne realizowanie kolejnych zadań był sam temat pracy, który okazał się interesujący dla członków grupy projektowej.

\section{Napotkane problemy}

Podczas realizacji projektu napotkano kilka problemów. Na samym początku prac problemem było zapoznanie się z~działaniem grafowej bazy danych, ponieważ żadna z~osób należąca do grupy projektowej bądź grupy TWO nie miała wcześniej styczności z~tego typu technologią. Ponadto dokumentacja bazy Neo4j nie jest zbyt szczegółowa, a~w~Internecie ciężko jest uzyskać fachową pomoc, dotyczącą zrozumienia mechanizmów jej działania.

Kolejnym napotkanym problemem było pozyskanie dostępu do API serwisu, z~którym można było się integrować. Prośby o~udostępnienie API zostały rozesłane do następujących portali:
\begin{itemize}
\item Skapiec.pl
\item Ceneo.pl
\item Nokaut.pl
\end{itemize}
Dopiero nokaut.pl po dwóch tygodniach oczekiwania udostępnił dostęp do swoich usług umożliwiających pozyskanie niezbędnych danych do projektu. Pozostałe serwisy udostępniały je odpłatnie lub w~celach komercyjnych, co sprawiło, że integracja z~tymi serwisami nie została zrealizowana.

\section{Rozbudowa projektu}

Aplikacja została stworzona w~taki sposób, by możliwa była jej dalsza rozbudowa. Klasy zaimplementowane w serwerowej części aplikacji umożliwiają łatwe dodawanie kolejnych komponentów oraz modułów do serwisu. Zaimplementowana warstwa komunikacyjna jest wystarczająco przejrzysta oraz zrozumiała, by w~łatwy sposób można było w przyszłości stworzyć nowy interfejs użytkownika komunikujący się z~aplikacją serwerową. Dzięki temu w~przyszłości można poszerzyć projekt o~wersję mobilną.
Istniejący interfejs użytkownika stworzony w~platformie ExtJs również został zaimplementowany w~sposób łatwy do rozbudowy.

\section{Wnioski dotyczące projektu}

Stworzona aplikacja udostępnia podstawowe funkcjonalności niezbędne do działania tego typu serwisów. Dodatkowo jest ona funkcjonalna i~z~pewnością mogłaby konkurować z~serwisami o podobnej tematyce. 
Prace nad projektem były przemyślane oraz dobrze zorganizowane. Umożliwiły one członkom grupy projektowej poszerzenie swojej wiedzy o~najnowsze technologie wykorzystywane do tworzenia tak rozbudowanych aplikacji. Powiększyli oni również swoje doświadczenie związane z~pracą grupową. 
Na pewno tego typu doświadczenia przydadzą się każdemu uczestnikowi projektu w~jego dalszej karierze zawodowej.

Jeśli grupa projektowa miałaby za zadanie stworzyć na nowo taki sam system to na pewno praca przebiegałaby trochę inaczej. Przede wszystkim zastosowano by wyraźniejszy podział pracy wśród członków. Każdy posiadałby odrębny zakres pracy. Pierwsza osoba miałaby przydzielane zadania związane z~aplikacją kliencką, druga z~częścią serwerową i~bazodanową, kolejna dotyczące algorytmu rankingowego, a~ostatnia związane z integracją systemu z API udostępnianym przez zewnętrzne serwisy. Taki podział obowiązków prawdopodobnie usprawnił by pracę nad projektem, a~sama aplikacja stałaby się jeszcze bardziej dopracowana oraz modularna. 