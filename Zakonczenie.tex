\chapter{Zakończenie}

\section{Realizacja tematu}

W ramach projektu udało się zrealizować wszystkie zamierzone cele. Wymaganiami określonymi przez grupę TWO były:
\begin{itemize}
\item Moduł społecznościowy
\item Moduł informacji o produktach
\item Moduł komunikacji z zewnętrznym serwisem
\item Moduł wyceny wiarygodności
\item Moduł administracyjny
\end{itemize}
Rezultat ten został osiągnięty dzięki dobrze zorganizowanej pracy nad projektem oraz wsparciu grupy TWO i promotora. Dodatkowym czynnikiem wpływającym na sprawne realizowanie kolejnych zadań było zainteresowanie samym tematem pracy, który jest ciekawy.

\section{Napotkane problemy}

Podczas realizacji projektu napotkano na kilka problemów. Na samym początku prac problemem było zapoznanie się z działaniem grafowej bazy danych, ponieważ żadna z osób należąca do grupy projektowej jak i grupy TWO nie miała wcześniej styczności z tego typu bazą danych. Ponadto dokumentacja tej bazy danych nie jest zbyt szczegółowa, a także w sieci ciężko jest uzyskać fachową pomoc dotyczącą zrozumienia mechanizmów jej działania.
Kolejnym napotkanym problemem było pozyskanie dostępu do API serwisu, z którym można by się integrować. Prośby o udostępnienie API zostały rozesłane do następujących portali:
\begin{itemize}
\item Skapiec.pl
\item Ceneo.pl
\item Nokaut.pl
\end{itemize}
Dopiero nokaut.pl po dwóch tygodniach oczekiwania udostępnił dostęp do swoich usług umożliwiających pozyskanie niezbędnych danych do projektu. Pozostałe serwisy udostępniały je odpłatnie lub w celach komercyjnych, co sprawiło, że integracja z tymi serwisami nieostała zrealizowana.

\section{Rozbudowa projektu}

Aplikacja została stworzona w taki sposób by możliwa była jej dalsza rozbudowa. Zaimplementowane klasy po stronie aplikacji serwerowej są na tyle generyczne, że umożliwiają łatwe dodawanie kolejnych komponentów oraz modułów do serwisu. Zaimplementowana warstwa komunikacyjna jest na tyle przejrzysta oraz zrozumiała, że w łatwy sposób można stworzyć kolejny interfejs użytkownika komunikującego się z aplikacją serwerową. Dzięki temu w przyszłości można by poszerzyć projekt o wersję mobilną.
Istniejący interfejs użytkownika stworzony w platformie ExtJs również jest zaimplementowany w sposób łatwy do rozbudowy. 

\section{Wnioski dotyczące projektu}

Stworzona aplikacja udostępnia podstawowe funkcjonalności niezbędne do działania tego typu serwisów. Pocieszającym aspektem jest to że aplikacja jest funkcjonalna i z pewnością mogła by konkurować z serwisami o podobnej tematyce. 
Prace nad projektem były przemyślane oraz dobrze zorganizowane. Umożliwiły one członkom grupy projektowej poszerzenie swojej wiedzy o najnowsze technologie wykorzystywane do tworzenia tak rozbudowanych aplikacji. Poszerzyli oni również swoje doświadczenie związane z pracą grupową. 
Na pewno tego typu doświadczenia przydadzą się każdemu uczestnikowi projektu w jego dalszej karierze zawodowej.