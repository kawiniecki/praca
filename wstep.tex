\chapter{Wstęp}

\section{Przedstawienie problemu}

W chwili obecnej rynek konsumencki oferuje nieograniczoną liczbę różnego rodzaju produktów. Różni producenci wytwarzają towary zróżnicowane nie tylko pod względem cenowym ale także jakościowym. Często okazuje się, że jakość towarów nie idzie w parze z ceną. Zdarza się, że niektórzy wytwórcy  zawyżają ceny oferowanych dóbr, przez co niemal identyczne produkty w konkurencyjnej firmie można kupić po o wiele korzystniejszej cenie. Normalne jest więc, że klient przed zakupem stara się zebrać jak najwięcej informacji o interesującym go produkcie. Nie każdemu wystarcza już tylko cena, dane techniczne czy skład danego towaru.  Coraz częściej szuka on opinii od osób, które miały już styczność z interesującym produktem i potrafią wskazać zarówno jego wady jak i zalety. 
\paragraph{}
W takiej sytuacji z pomocą przychodzą serwisy internetowe umożliwiające porównywanie cen, produkt pochodzących z różnych sklepów internetowych. Serwisy te na bieżąco aktualizują swoją bazę danych produktów i dzięki temu użytkownik ma pewność, że porównywarki zawierają aktualne dane. Dużą zaletą tych aplikacji jest możliwość informowania użytkownika o nowych promocjach czy wyprzedażach. Po pewnym czasie ten zestaw informacji nie był wystarczający dla przeciętnego konsumenta. Aby spełnić wymagania klientów serwisy te rozszerzyły swoją funkcjonalność o możliwość opiniowania przez kupujących nie tylko produktów ale także samych sklepów, jakości obsługi czy szybkości wysyłki. To sprawiło, iż stały się one niezwykle użyteczne i popularne wśród kupujących. Każdy użytkownik ma możliwość porównania opinii o wybranych produktach i wybrania tych, które są uznawane jako najlepsze przez innych użytkowników. Pojawił się jednak kolejny problem. Zaczęto zauważać, że opnie są coraz częściej przekłamane. Coraz więcej pojawiało się oszustów, którzy sztucznie promowali produkty, zachwalając je. W odpowiedzi na tego typu działania serwisy dodały możliwość oceniania recenzji, co miało na celu wyeliminowanie sprzecznych z prawdą opinii. Obecne jednak serwisy nie posiadają na tyle dobrych i zaawansowanych algorytmów, które potrafiły by stworzyć ranking wiarygodnych recenzji dotyczących produktów. Jak temu zaradzić?  


\section{Cel i zakres pracy}

Głównym celem projektu RevCommunity było stworzenie wiarygodnego systemu społecznościowego opinii o produktach. Aplikacja ma za zadanie pomóc użytkownikom przy zakupie określonego produktu po przez zapewnienie mu wiarygodnych i użytecznych opinii. Najważniejsze zadanie to opracowanie i użycie algorytmu rankingowego dla recenzji produktów. Główny cel to też możliwość dodawania recenzji dla danego produktu przez zalogowanych użytkowników. Każdy zalogowany użytkownik powinien też mieć możliwość ocenienia zarówno opinii innych użytkowników jak i samych produktów. Serwis ma umożliwić także możliwość wyszukiwania produktów według kryteriów użytkownika. Dodatkowo administrator systemu może w każdej chwili dodać nową kategorię produktów w raz z jej parametrami czy dodać nowy produkt do wybranej kategorii. W panelu administracyjnym ma być możliwość uruchomienia importu produktów wraz z opisami i cenami z zewnętrznego serwisu. Aplikacja powinna mieć też mechanizm subskrypcji produktów co ułatwiłoby użytkownikom śledzenie na bieżąco nowych produktów. Dodatkową funkcją jest moduł newslettera, który ma za zadanie okresowo wysyłać na maila informację o nowych recenzjach, produktach i subskrypcjach. 



\section{Podział pracy}
\noindent\begin{tabular}{rp{9cm}}
\textbf{Dawid Janaszak} & Interfejs użytkownika, algorytm rankingowy.\\

\textbf{Marcin Kaźmierski} & Interfejs użytkownika, moduł newsletter.\\

\textbf{Paweł Rosolak} & Interfejs użytkownika, model bazy danych, architektura aplikacji.\\

\textbf{Tomasz Straszewski} & Interfejs użytkownika, warstwa zabiezpieczeń, integracja z zewnętrzynymi serwisami.\\
\end{tabular}

