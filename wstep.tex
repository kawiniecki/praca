\chapter{Wstęp}

\section{Przedstawienie problemu}
W chwili obecnej rynek konsumencki oferuje praktycznie nieograniczoną liczbę różnego rodzaju produktów. Różni producenci wytwarzają towary zróżnicowane nie tylko pod względem cenowym, ale także jakościowym. Często okazuje się, że jakość towarów nie idzie w~parze z~ceną. Zdarza się, że niektórzy wytwórcy  zawyżają ceny oferowanych dóbr, przez co niemal identyczne produkty w~konkurencyjnej firmie można kupić po o~wiele korzystniejszej cenie. Normalne jest więc, że klient przed zakupem stara się zebrać jak najwięcej informacji o~interesującym go produkcie. Nie każdemu wystarcza już tylko cena, dane techniczne czy skład danego towaru.  Coraz częściej szuka on opinii osób, które miały już styczność z~interesującym produktem i~potrafią wskazać zarówno jego wady, jak i~zalety. 

W sytuacji, kiedy Internet nie był jeszcze tak mocno rozbudowany jak w~dniu dzisiejszym, z~pomocą przychodziły czasopisma konsumenckie. To w~nich właśnie redaktorzy opisywali testowane produkty. Opinia redaktorów dotycząca testowanych produktów wpływała na ich popyt wśród konsumentów. Konsument czytający recenzję nie mógł mieć jednak pewności czy przedstawiona opinia jest wiarygodna. Z~czasem, kiedy Internet zaczął odgrywać większą rolę, pojawiły się liczne fora, na których użytkownicy chętnie wypowiadali się na temat wad i~zalet produktów, z~którymi mieli styczność. Powstało sporo kanałów w~serwisie YouTube dotyczących opiniowania i~testowania produktów. Krótkometrażowe filmy udostępniane w~tym serwisie stanowiły wideo recenzję produktu.  Tego typu recenzja przedstawiała zarówno wizualne, jak i~funkcjonalne walory produktu. 

Natenczas ("Nateczas" znaczy mniej wiecej tyle co "wówczas", "wtedy". Moim zdaniem pospolite "Obecnie" będzie bardziej odpowiednie w tym miejscu.) najczęściej odwiedzane przez ogromne grono kupujących są serwisy internetowe, które umożliwiają porównywanie cen i~produktów pochodzących z~różnych sklepów internetowych. Serwisy te na bieżąco aktualizują swoją bazę danych produktów i~dzięki temu użytkownik ma pewność, że zawierają one aktualne dane. Dużą zaletą tych aplikacji jest możliwość informowania użytkownika o~nowych promocjach czy wyprzedażach. Jednak po pewnym czasie taki zestaw informacji stał się niewystarczający dla przeciętnego konsumenta. Aby spełnić wymagania klientów serwisy rozszerzyły swoją funkcjonalność o~możliwość opiniowania przez kupujących nie tylko produktów, ale także samych sklepów, jakości obsługi czy szybkości wysyłki. To sprawiło, iż stały się one niezwykle użyteczne i~popularne wśród kupujących. Każdy użytkownik miał możliwość porównania opinii o~wybranych produktach i~wybrania tych, które są uznawane za najlepsze przez innych użytkowników. Pojawił się jednak kolejny problem. Zaczęto zauważać, że opinie coraz częściej były przekłamane. Coraz więcej pojawiało się oszustów, którzy zachwalając produkty sztucznie je promowali. W odpowiedzi na tego typu działania serwisy dodały możliwość oceniania recenzji, co miało na celu wyeliminowanie sprzecznych z~prawdą opinii. Obecnie jednak, opisane wyżej serwisy nie posiadają na tyle skutecznych algorytmów, które potrafiłyby stworzyć ranking wiarygodnych recenzji dotyczących produktów. Jak temu zaradzić?  
 


\section{Cel i~zakres pracy}
Głównym celem projektu RevCommunity było stworzenie wiarygodnego społecznościowego systemu  opinii o~produktach. Aplikacja ma za zadanie pomóc użytkownikom przy zakupie określonego produktu poprzez zapewnienie mu dostępu do wiarygodnych i~użytecznych opinii. Najważniejsze zadanie to opracowanie i zastosowanie algorytmu rankingowego dla recenzji produktów. Głównym celem jest stworzenie możliwości dodawania recenzji dla danego produktu przez zalogowanych użytkowników. Każdy zalogowany użytkownik powinien też mieć możliwość ocenienia zarówno opinii innych użytkowników, jak i~samych produktów. Serwis ma umożliwić także wyszukiwanie produktów według kryteriów użytkownika. Dodatkowo administrator systemu będzie mógł w~każdej chwili dodać nową kategorię produktów wraz z~jej parametrami, czy dodać nowy produkt do wybranej kategorii. W panelu administracyjnym ma być możliwość uruchomienia importu produktów wraz z~opisami i~cenami z~zewnętrznego serwisu. Aplikacja powinna mieć także mechanizm subskrypcji produktów, co ułatwiłoby użytkownikom śledzenie na bieżąco nowych produktów. Dodatkową funkcją będzie moduł newslettera, którego zadanie będzie polegało na okresowym wysyłaniu na maila informacji o~nowych recenzjach, produktach i subskrypcjach. 

\section{Układ pracy}
Niniejsza praca składa się z~9 rozdziałów. Wstęp zawiera motywacje, które wpłynęły na realizacją projektu. Rozdział „Podstawy teoretyczne” przedstawia istniejące, podobne rozwiązania na rynku, opis grafowej bazy danych oraz teoretyczne podstawy algorytmu rankingowego. W rozdziale „Specyfikacja” przedstawiono wymagania techniczne, które powinien spełniać projekt. „Organizacja pracy” opisuje sposób komunikacji pomiędzy wszystkimi osobami zaangażowanymi w~tworzenie projektu.  Rozdział „Algorytm rankingowy” w~całości poświęcony jest idei algorytmu rankingowego recenzji. W rozdziale „Część serwerowa aplikacji” przedstawiono architekturę systemu po stronie serwera, natomiast w~rozdziale „Część kliencka aplikacji” została opisana architektura po stronie klienta. W rozdziale „Przykładowe wykorzystanie systemu” przedstawiono najważniejsze funkcje i~możliwości systemu. Podsumowanie całej pracy, zawierające wnioski oraz napotkane problemy podczas pracy nad projektem, zawarte zostały w~rozdziale „Zakończenie”.


\section{Podział pracy}
\noindent\begin{tabular}{rp{9cm}}
\textbf{Dawid Janaszak} & Interfejs użytkownika, algorytm rankingowy.\\

\textbf{Marcin Kaźmierski} & Interfejs użytkownika, moduł newsletter.\\

\textbf{Paweł Rosolak} & Interfejs użytkownika, model bazy danych, architektura aplikacji.\\

\textbf{Tomasz Straszewski} & Interfejs użytkownika, warstwa zabiezpieczeń, integracja z~zewnętrzynymi serwisami.\\
\end{tabular}

